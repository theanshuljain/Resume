\documentclass[a4paper,9pt]{article}

\usepackage{latexsym}
\usepackage[empty]{fullpage}
\usepackage{titlesec}
\usepackage{xcolor}
\usepackage{marvosym}
\usepackage{verbatim}
\usepackage{enumitem}
\usepackage[hidelinks]{hyperref}
\usepackage{fancyhdr}
\usepackage{multicol}
\usepackage{graphicx}
\usepackage{fontawesome}
\usepackage{ragged2e}

\pagestyle{fancy}
\fancyhf{}
\fancyfoot{}
\renewcommand{\headrulewidth}{0pt}
\renewcommand{\footrulewidth}{0pt}

\definecolor{darkgreen}{rgb}{0.0, 0.5, 0.0}

\addtolength{\oddsidemargin}{-0.65in}
\addtolength{\evensidemargin}{-0.65in}
\addtolength{\textwidth}{1.3in}
\addtolength{\topmargin}{-0.6in}
\addtolength{\textheight}{1.3in}

\linespread{0.9}
\urlstyle{rm}
\raggedbottom
\raggedright
\setlength{\tabcolsep}{0in}

% Section formatting
\titleformat{\section}{
    \vspace{-10pt}\scshape\raggedright\large
}{}{0em}{}[\color{black}\titlerule \vspace{-6pt}]

%----------------------------------
% Custom commands
\newcommand{\resumeItem}[2]{
    \item\small{
        \textbf{#1}{: #2 \vspace{-2pt}}
    }
}

\newcommand{\resumeItemWithoutTitle}[1]{
    \item\small{
        #1 \vspace{-2pt}
    }
}

\newcommand{\resumeSubheading}[4]{
    \vspace{-1pt}\item
        \begin{tabular*}{0.97\textwidth}{l@{\extracolsep{\fill}}r}
            \textbf{#1} & #2 \\
            \textit{#3} & \textit{#4} \\
        \end{tabular*}\vspace{-5pt}
}

\renewcommand{\labelitemii}{$\circ$}
\newcommand{\resumeSubHeadingListStart}{\begin{itemize}[leftmargin=*]}
\newcommand{\resumeSubHeadingListEnd}{\end{itemize}}
\newcommand{\resumeItemListStart}{\begin{itemize}[leftmargin=*]}
\newcommand{\resumeItemListEnd}{\end{itemize}\vspace{-5pt}}

%-----------------------------------------
% CV Starts

\begin{document}

%-----------------------------------------
% Heading

\begin{center}
    \textbf{{\LARGE Anshul Jain}} \\
    \vspace{3pt}
    \scriptsize{
        \faLinkedin \hspace{2pt} \textcolor{blue}
        {\href{https://www.linkedin.com/in/theanshuljain/}{anshul-jain}} \hspace{5pt}
        \faGithub \hspace{2pt} \textcolor{blue}
        {\href{https://github.com/theanshuljain}{Anshul-Jain}} \hspace{5pt}
        \faEnvelope \hspace{2pt} \textcolor{blue}{\href{mailto:anshuljain6598@gmail.com}{anshuljain6598@gmail.com}} \hspace{5pt}
        \faPhone \hspace{2pt} +1 (206) 600-1808 \hspace{5pt}
        \faMapMarker \hspace{2pt} Boulder, CO, USA
  }
\end{center}

%------------------------------------------
% Objective

\section{Objective}
\begin{justify}
Driven and hardworking professional with a strong foundation in Engineering Mathematics, Physics, and System Design. As a Graduate Aerospace Engineer and passionate educator, I am eager to take on new challenges that allow me to utilize my skills in strategic planning, training and development, relationship building, and project management. Proven ability to lead cross-functional teams, mentor peers, and simplify complex technical concepts into accessible lessons and hands-on demonstrations. Committed to showing up as my authentic self, following my passions, and adding meaningful value to every role I undertake. Seeking teaching opportunities where I can use my background and expertise to inspire curiosity, analytical thinking, and a lifelong love for learning in students.
\end{justify}

%------------------------------------------
% Education

\section{Education}
\resumeSubHeadingListStart

\resumeSubheading
{University of Colorado Boulder}{Aug 2023 – Dec 2025}
{Master’s in Aerospace Engineering Sciences}{Boulder, CO}

\resumeSubheading
{R. V. College of Engineering}{Aug 2019 – Apr 2023}
{Bachelor’s in Aerospace Engineering}{Bengaluru, India}

\resumeSubHeadingListEnd

\textbf{Programming and Tools:} MATLAB, Simulink, Python, C, C++, JavaScript, R, VS Code, Git, GitHub, Canvas.

%-------------------------------------------
% Core Competencies
\section{Core Competencies}
\resumeItemListStart
\resumeItemWithoutTitle{\textbf{Mathematics:} Calculus, Linear Algebra, Matrix Algebra, Eigenvalue Analysis, Differential Equations, State-Space Modeling, System Identification, Curve Fitting, Transfer Function Analysis, Optimization, LQR and Feedback Design, Probability, Statistics, Gaussian Transformations, and Numerical Simulation.}
\resumeItemWithoutTitle{\textbf{Physics:} Rigid-Body Dynamics, Rotational and Translational Motion, Torque Analysis, Fluid Resistance, Aerodynamics, Control Systems, Frequency-Domain Methods (Bode, Nyquist, Stability Margins), and System Stability Analysis.}
\resumeItemWithoutTitle{\textbf{Instructional Skills:} Curriculum Planning, Lesson Preparation, Classroom Facilitation, Student Guidance, Academic Support, and Educational Communication.}
\resumeItemWithoutTitle{\textbf{Technical Tools:} MATLAB, Simulink, Python, C, C++, JavaScript, Git, VS Code, Canvas.}
\resumeItemListEnd

%-------------------------------------------
% Core Competencies
%\section{Core Competencies}

%\textbf{Mathematics:} Calculus, Differential Equations, Linear Algebra, State-Space Modeling, System Identification, Optimization, Probability and Statistics, Numerical Simulation.
%\textbf{Physics:} Mechanics (Newtonian, Rotational, Fluid), Dynamics, Aerodynamics, Control Systems, System Stability, and Feedback Analysis.
%\textbf{Instructional Skills:} Lesson and Lab Development, Concept Simplification, Student Mentoring, Performance Assessment, and Scientific Communication.
%\textbf{Technical Tools:} MATLAB, Simulink, Python, C++, Git, and VS Code.

%-------------------------------------------
% Mathematics and Physics Expertise
%\section{Mathematics and Physics Expertise}
%\resumeItemListStart
%\resumeItemWithoutTitle{Linear Algebra, Matrix Algebra, and Eigenvalue Analysis}
%\resumeItemWithoutTitle{Differential Equations and State-Space Modeling}
%\resumeItemWithoutTitle{System Identification, Curve Fitting, and Transfer Function Analysis}
%\resumeItemWithoutTitle{Frequency-Domain Methods: Bode, Nyquist, and Stability Margins}
%\resumeItemWithoutTitle{Optimization and Control Design using LQR and Feedback Principles}
%\resumeItemWithoutTitle{Probability, Statistics, and Gaussian Transformations}
%\resumeItemWithoutTitle{Numerical Simulation and Data Interpretation using MATLAB and Python}
%\resumeItemWithoutTitle{Rigid-Body Dynamics, Torque Analysis, and Fluid Resistance Principles}
%\resumeItemListEnd

%-------------------------------------------
% Teaching and Mentoring Experience
\section{Teaching Experience}
\resumeSubHeadingListStart

\resumeSubheading
{Teaching Assistant}{Aug 2023 – Oct 2025}
{University of Colorado Boulder}{Boulder, CO}
\resumeItemListStart
\resumeItemWithoutTitle{Taught undergraduate laboratories including \textbf{EBIO 1110: Biology and Society}, and \textbf{EBIO 1230/1240: General Biology}, guiding 50+ students through experiments that connected theoretical and real-world biological and mathematical principles.}
\resumeItemWithoutTitle{Developed and implemented lesson plans, lab manuals, and problem sets that simplified complex concepts into intuitive, hands-on learning experiences.}
\resumeItemWithoutTitle{Trained students on lab safety procedures and proper operation of biological laboratory equipment, ensuring a safe and productive environment.}
\resumeItemWithoutTitle{Communicated regularly with students and faculty to coordinate testing accommodations, address course-related concerns, and resolve academic or logistical issues.}
\resumeItemWithoutTitle{Maintained and updated course Canvas pages, announcements, and discussion boards to build a strong virtual learning presence and improve student engagement.}
\resumeItemWithoutTitle{Monitored student performance and participation through grade tracking and feedback systems, identifying learners needing additional support or guidance.}
\resumeItemWithoutTitle{Gathered student feedback through mid-semester check-ins and end-of-term evaluations to assess teaching effectiveness and improve course delivery.}
\resumeItemWithoutTitle{Collaborated with faculty and departmental staff to enhance instructional quality, ensure accessibility compliance, and promote a positive learning culture.}
\resumeItemListEnd

\resumeSubheading
{Mentor, Recovery Subsystem Engineer}{Oct 2019 – Aug 2022}
{Team Antariksh, R. V. College of Engineering}{Bengaluru, India}
\resumeItemListStart
\resumeItemWithoutTitle{Mentored junior team members in the principles of \textbf{aerodynamics, fluid mechanics, and system modeling} by connecting theory with real-world rocketry applications.}
\resumeItemWithoutTitle{Guided students through the design and analysis of \textbf{dual-parachute recovery systems}, explaining concepts such as drag force, lift generation, pressure distribution, and momentum transfer.}
\resumeItemWithoutTitle{Led workshops demonstrating how parameters like parachute diameter, mass, and air density affect terminal velocity and stability during descent.}
\resumeItemWithoutTitle{Supervised CFD and experimental tests to validate theoretical results, helping students interpret data and refine their understanding of physical principles.}
\resumeItemWithoutTitle{Developed visual explanations and calculation templates to enhance comprehension of recovery dynamics and stability criteria.}
\resumeItemWithoutTitle{Derived analytical relationships between drag coefficient and Reynolds number to assess performance under varying flow regimes.}
\resumeItemWithoutTitle{Documented design iterations, test results, and analytical findings in technical reports to support subsystem validation and team knowledge retention.}
\resumeItemWithoutTitle{Encouraged teamwork and scientific reasoning by fostering a collaborative environment focused on applying physics to practical engineering problems.}
\resumeItemListEnd

\resumeSubHeadingListEnd

%-------------------------------------------
% Academic Projects
\section{Projects}
\resumeSubHeadingListStart

\resumeSubheading
{Automatic Control Systems – Spacecraft Attitude Control Projects}{Jan 2024 – May 2024}
{University of Colorado Boulder}{Boulder, CO}
\resumeItemListStart
\resumeItemWithoutTitle{Analyzed spacecraft mockup frequency-response data to derive mathematical models representing rotational dynamics through curve fitting, transfer-function formulation, and Bode/Nyquist interpretation.}
\resumeItemWithoutTitle{Translated frequency-domain characteristics into state-space representation to study system stability, controllability, and observability using matrix algebra.}
\resumeItemWithoutTitle{Designed and optimized feedback and observer controllers via pole placement and LQR principles, quantifying system behavior through eigenvalue analysis, damping ratios, and bandwidth metrics.}
\resumeItemWithoutTitle{Simulated time-domain responses under step and sinusoidal inputs to illustrate control trade-offs between response speed, actuator saturation, and dynamic stability.}
\resumeItemListEnd

\resumeSubheading
{Attitude Determination and Control Module}{Aug 2023 – May 2024}
{University of Colorado Boulder}{Boulder, CO}
\resumeItemListStart
\resumeItemWithoutTitle{Applied principles of rotational dynamics and rigid-body kinematics to design a CubeSat attitude control system meeting sub-arcsecond pointing requirements.}
\resumeItemWithoutTitle{Derived and implemented mathematical models based on Euler’s equations of motion and feedback stability analysis to evaluate control performance.}
\resumeItemWithoutTitle{Developed a Digital Sun Sensor model in C++ using the Basilisk framework to simulate sensor-based attitude estimation under realistic illumination conditions.}
\resumeItemWithoutTitle{Collaborated with team members to analyze inertia tensor effects, control torque interactions, and system stability trade-offs through simulation-driven discussions.}
\resumeItemListEnd

\resumeSubheading
{Speed Control of a Brushless DC Motor using LQR}{May 2022 – June 2022}
{R. V. College of Engineering}{Bengaluru, India}
\resumeItemListStart
\resumeItemWithoutTitle{Formulated the dynamic model of a brushless DC motor using differential equations representing voltage–current and torque–speed relationships.}
\resumeItemWithoutTitle{Designed and implemented a Linear Quadratic Regulator (LQR) in MATLAB to minimize state error and control effort under varying load conditions.}
\resumeItemWithoutTitle{Analyzed system stability, transient response, and steady-state accuracy through eigenvalue placement and performance metric evaluation.}
\resumeItemWithoutTitle{Collaborated with peers to interpret the effects of feedback gain tuning on overall control efficiency and physical system behavior.}
\resumeItemListEnd

\resumeSubheading
{Mathematical Image Modeling}{May 2022 – June 2022}
{R. V. College of Engineering}{Bengaluru, India}
\resumeItemListStart
\resumeItemWithoutTitle{Developed mathematical models in MATLAB to animate images by transforming pixel brightness distributions using Gaussian functions.}
\resumeItemWithoutTitle{Applied concepts of probability, normalization, and curve fitting to analyze the effect of mean and variance on image intensity patterns.}
\resumeItemWithoutTitle{Explored correlations between spatial frequency content and visual perception to interpret image transformations mathematically.}
\resumeItemWithoutTitle{Collaborated with peers to evaluate numerical results and optimize model parameters for smoother intensity transitions.}
\resumeItemListEnd

\resumeSubHeadingListEnd

%-------------------------------------------
% Research Experience
\section{Research Experience}
\resumeSubHeadingListStart
\resumeSubheading
{Co-Author, 73rd International Astronautical Congress}{Nov 2021 – Sept 2022}
{Team Antariksh}{Bengaluru, India}
\resumeItemListStart
\resumeItemWithoutTitle{Vageesha S, Darpan B, Trisha A, Anshul Jain, Greeshma A, Rithwik R, “Study of Drag Characteristics of a Parachute for landing on planets and moons with different atmospheric conditions and its optimization using gases with varying properties”.}
\resumeItemListEnd
\resumeSubHeadingListEnd

%-------------------------------------------
% Awards and Recognition
\section{Awards and Recognition}
\resumeSubHeadingListStart

\resumeSubheading
{SAARTHAKA Trust Scholarship}{Jan 2020}
{R. V. College of Engineering}{Bengaluru, India}
\resumeItemListStart
\resumeItemWithoutTitle{Awarded for securing \textbf{Rank 1 in Aerospace Engineering (2019)} among all first-year students.}
\resumeItemListEnd

\resumeSubheading
{E-Summit Business Marathon – 1st Place}{July 2022}
{R. V. College of Engineering}{Bengaluru, India}
\resumeItemListStart
\resumeItemWithoutTitle{Ranked \textbf{1st out of 32 teams} for developing a service-based transport solution addressing damaged road infrastructure in hilly regions during monsoon conditions.}
\resumeItemListEnd

\resumeSubHeadingListEnd

%-------------------------------------------
% Volunteer Experience
\section{Volunteer Experience}
\resumeSubHeadingListStart

\resumeSubheading
{Emcee – Faculty Development Program (FDP)}{May 2022}
{R. V. College of Engineering}{Bengaluru, India}
\resumeItemListStart
\resumeItemWithoutTitle{Coordinated a 5-day FDP on Machine Learning applications in Aerospace Engineering, facilitating sessions, managing logistics, and ensuring smooth communication between faculty and participants.}
\resumeItemListEnd

\resumeSubheading
{Emcee – RSS Vishwa Sangh Shibir}{Dec 2015}
{Rashtriya Swayamsevak Sangh}{Indore, India}
\resumeItemListStart
\resumeItemWithoutTitle{Served as the Tamil-language Emcee for the international conference, engaging with delegates from 45+ countries after rapidly acquiring conversational proficiency in the language.}
\resumeItemListEnd

\resumeSubHeadingListEnd

%-------------------------------------------
% Languages
\section{Languages}
\resumeItemListStart
\resumeItemWithoutTitle{English: Professional Proficiency}
\resumeItemWithoutTitle{Hindi: Native Proficiency}
\resumeItemListEnd

%-------------------------------------------
\end{document}